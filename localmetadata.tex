\title{Advances \newlineCover in the study of \newlineCover Siouan languages \newlineCover and linguistics} 
%\subtitle{Change your subtitle in localmetadata.tex}
\BackTitle{Advances  in the study of  Siouan languages and linguistics}
\BackBody{The Siouan family comprises some twenty languages, historically spoken across a broad swath of the central North American plains and woodlands, as well as in parts of the southeastern United States.
In spite of its geographical extent and diversity, and the size and importance of several Siouan-speaking tribes, this family has received relatively little attention in the linguistic literature and many of the individual Siouan languages are severely understudied. This volume aims to make work on Siouan languages more broadly available and to encourage deeper investigation of the myriad typological, theoretical, descriptive, and pedagogical issues they raise.

The 17 chapters in this volume present a broad range of current Siouan research, focusing on various Siouan languages, from a variety of linguistic perspectives: historical-genetic, philological, applied, descriptive, formal/generative, and comparative/typological. The editors' preface summarizes characteristic features of the Siouan family, including head-final and ``verb-centered" syntax, a complex system of verbal affixes including applicatives and subject-possessives, head-internal relative clauses, gendered speech markers, stop-systems including ejectives, and a preference for certain prosodic and phonotactic patterns.

The volume is dedicated to the memory of Professor Robert L. Rankin, a towering figure in Siouan linguistics throughout his long career, who passed away in February of 2014.}
\dedication{To Bob, whose knowledge was matched only by his generosity.}
\typesetter{%
Bryan James Gordon,
Sebastian Nordhoff,
Catherine Rudin
}
\proofreader{%
Aaron Sonnenschein,
Alec Shaw,
Alessia	Battisti,
Andreas Hölzl,
Aneilia Stefanova,
Carolyn	O'Meara,
Christian Döhler,
Dominik	Luke\v{s},	 
Eitan Grossman,	
Elizabeth Zeitoun,
Matthew	Czuba,
Martin Haspelmath,	
Neal Whitman,
Roelant Ossewaarde,
Rong Chen,	
Stathis	Selimis,
Steve	Pepper,
Teresa	Proto,
Varun de Castro Arrazola	
}
\illustrator{Sebastian Nordhoff}
\author{Catherine Rudin\lastand  Bryan J. Gordon}
% \author{{Edited by}\newlineCover Catherine Rudin\newlineCover Bryan J. Gordon}
\SpineAuthor{Rudin \& Gordon (eds.)\hspace*{-1cm}}
\SpineTitle{\hspace*{-1cm}Advances in the study of {Siouan} languages and linguistics\hspace*{-1cm}}
\renewcommand{\lsISBNdigital}{978-3-946234-37-1}
\renewcommand{\lsISBNhardcover}{978-3-946234-38-8}
% % % \renewcommand{\lsISBNsoftcover}{978-3-946234-39-5} % discontinued
% % % \renewcommand{\lsISBNsoftcoverus}{978-1-530465-99-6} % discontinued
\renewcommand{\lsSeries}{sidl} % use lowercase acronym, e.g. sidl, eotms, tgdi
\renewcommand{\lsSeriesNumber}{10} %will be assigned when the book enters the proofreading stage
\renewcommand{\lsURL}{http://langsci-press.org/catalog/book/94} % contact the coordinator for the right number
% % \BookDOI{10.17169/langsci.b94.118} % The DOI of the 2016 original publication
\BookDOI{10.5281/zenodo.3741670} % The DOI of the 2020 version with minor improvements
